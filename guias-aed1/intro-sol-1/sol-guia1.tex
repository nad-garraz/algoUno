\documentclass[12pt,a4paper,spanish]{article}
% Sacar draft para que aparezcan las imagenes.
% Opciones: 10pt, 11pt, landscape, twocolumn, fleqn, leqno...
% Opciones de clase: article, report, letter, beamer...

% Paquetes:
% =========

\usepackage[headheight=110pt, top = 2cm, bottom = 2cm, left=2cm, right=1cm]{geometry} %modifico márgenes
\usepackage[T1]{fontenc} % Para tildes en espñol
\usepackage[utf8]{inputenc} % Para poder escribir con tildes en el editor.
\usepackage[spanish]{babel} % Para cortar las palabras en silabas, creo.
\usepackage[ddmmyyyy]{datetime}
\usepackage{amsmath} % Soporte de mathmatics
\usepackage{amssymb} % fuentes de mathmatics y símbolos
\usepackage{array} % Para tablas y eso
\usepackage{caption} % Configuracion de figuras y tablas
\usepackage{xcolor} % Para colorear el texto: black, blue, brown, cyan, darkgray, gray, green, lightgray, lime, magenta, olive, orange, pink, purple, red, teal, violet, white, yellow.
\usepackage{graphicx} % Necesario para poner imagenes
\usepackage{enumitem} % Cambiar labels y más flexibilidad para el enumerate
\usepackage{multicol} 
\usepackage{tikz} % para graficar
\usepackage{titlesec} % para poder modificar los titulos y secciones

\usepackage{bbding} % símbolos de donde uso
%FiveStar,

% Definiciones y nuevos comandos:
% =============

% Conjuntos
\def\reales{\mathbb R}
\def\naturales{\mathbb N}
\def\enteros{\mathbb Z}
\def\complejos{\mathbb C}
\def\i{\text{i}} % i^2 =  -1
\def\equivale{\equiv}

\def\vacio{\varnothing}
\def\union{\cup}
\def\inter{\cap}
\def\existe{\exists\,}
\def\noexiste{\nexists\,}
\def\paratodo{\forall}
\def\talque{\;|\;}
\newcommand{\set}[1] { \left\{ #1 \right\} } % Conjuntos entre llaves

% lógica proposicional
\def\neg{\sim}
\def\y{\land}
\def\o{\lor}
\def\entonces{\rightarrow}
\def\sisolosi{\leftrightarrow}

% lógica trivaluada 
\def\indef{\perp}
\def\no{\,\lnot\,}
\def\lo{\lor_{\text{L}}}
\def\ly{\,\land_{\text{L}}\,}
\def\lentonces{\rightarrow_{\text{L}}}

% =====
% Miscelanea
% =====
\newcommand{\estabien}{{\color{blue} Consultado, está bien. \checkmark}}
\newcommand{\hacer}{{\color{black!30!red}Hacer!}}

% separador
\def\separador{\noindent\rule{\linewidth}{0.4pt}\\}
\def\separadorCorto{\noindent\rule{0.5\linewidth}{0.4pt}\\}


% Llaves, paréntesis, contenedores
\newcommand{\llave}[2]{ \left\{ \begin{array}{#1} #2 \end{array}\right. }
\newcommand{\llaves}[2]{ \left\{ \begin{array}{#1} #2 \end{array} \right\} }
\newcommand{\matriz}[2]{\left( \begin{array}{#1} #2 \end{array} \right)}


% Colores
\newcommand{\red}[1]{ {\color{red} \text{#1}}}
\newcommand{\green}[1]{ {\color{olive} \text{#1}}}
\newcommand{\blue}[1]{ {\color{blue} \text{#1}}}
\newcommand{\magenta}[1]{ {\color{magenta} \text{#1}}}

% Stackrel text
\newcommand{\stacktext}[2]{ \stackrel{\text{#1}}{#2} }

% Flecha con texto
\NewDocumentCommand{\flecha}{m o}{
	\IfNoValueTF{#2}{
		\xrightarrow[]{\text{#1}}
	}{
		\xrightarrow[\text{#2}]{\text{#1}}
	}
}

% sección ejercicio con su respectivo formato y contador
\newcounter{ejercicio}[part]
\renewcommand{\theejercicio}{\arabic{ejercicio}}
\newcommand{\ejercicio}{%
	\stepcounter{ejercicio}%
	\titleformat{\subsubsection}[runin]{\normalfont\bfseries}{Ejercicio \theejercicio}{-1em}{}%
  \subsubsection*{Ejercicio \theejercicio.}%
}
	\let\oldsection\part
	\renewcommand{\part}{\stepcounter{ejercicio}\part}
 % idem con las definiciones

\begin{document}
\pagestyle{empty}
% Datos de importancia
\title{Práctica 1 de Intro}
\author{D. Garraz}
\date{Actualizado: \today} % Cambiar de ser necesario
\maketitle

\section{Lógica Binaria (Verdadero o Falso)}

\fbox{
\parbox{\textwidth}{
  \textbf{Fórmulas según teórica}:
  \begin{enumerate}
    \item True y False son fórmulas.
    \item Cualqueir variable proposicional es una fórmula.
    \item Si $A$ es una fórmula, $\no A$ es una fórmula.
    \item Si $A_1, A_2, \dots, A_n$ son fórmulas, ($A_1 \y A_2 \y \cdots \y A_n)$ es una fórmula.
    \item Si $A_1, A_2, \dots, A_n$ son fórmulas, ($A_1 \o A_2 \o \cdots \o A_n)$ es una fórmula.
    \item Si $A$ y $B$ son fórmulas, $(A \entonces B)$ es una fórmula.
    \item Si $A$ y $B$ son fórmulas, $(A \sisolosi B)$ es una fórmula.
  \end{enumerate}
}
}


\ejercicio \FiveStar\ Sean $p$ y $q$ variables proposicionales. Siguiendo las reglas de formación de fórmulas, ¿Cuánles de las siguientes expresiones son \textit{fórmulas bien formadas}?

\begin{enumerate}[label=\alph*)]
	\item $(p \no q)$\\
	      No es fórmula, si bien $p$ y $\no q$ son fórmulas. Esa forma de juntarlas no obedece a ninguna de las siete reglas.

	\item  $p \o q \o True$\\
	      No es fórmula, faltan paréntesis para saber como evaluar.

	\item $(p \entonces \no p \entonces q)$\\
	      No es fórmula,. Si bien se parece a una extensión de la regla 6, faltarían los paréntesis, ejemplo: $( (p \entonces \no p) \entonces q)$ o $( p \entonces (\no p \entonces q))$

	\item $\no (p)$\\
	      Dudoso. Diría que \textbf{no} es una fórmula, porque el agregar paréntesis al pedo no lo tenemos definido en la lista de reglas.

	\item $p \o \no p \y q)$
	      No es fórmula, se parece a una mezcla de las reglas 4 y 5 pero faltan paréntesis onda $((p \o \no p) \y q)$ o  $(p \o (\no p \y q))$\\
	      \red{¿Mejores palabras para describir lo que pasa?}

	\item $(True \y True \y True)$\\
	      Es fórmula por regla 4.
\end{enumerate}

\ejercicio\FiveStar\ 
Determinar el valor de verdad de las siguiente fórmulas:

\begin{enumerate}[label=\arabic*. ]
	\item Cuando el valor de verdad de a, b y c es verdadero, mientras que el de x e y es falso.
	\item Cuando el valor de verdad de a, b y c es falso, mientras que el de x e y es verdadero.
\end{enumerate}

\begin{multicols}{2}
	\begin{enumerate}[label=\alph*)]
		\item $(\no a \o b) $
		      \begin{enumerate}[label=\arabic*. ]
			      \item V
			      \item V
		      \end{enumerate}
		\item $(c \o (y \y x) \o b)$
		      \begin{enumerate}[label=\arabic*. ]
			      \item V
			      \item V
		      \end{enumerate}
		\item $\no (c \o y)$ \hacer
		\item \hacer
		\item \hacer
		\item $(((c \o y) \y (x \o b)) \sisolosi (c \o (y \y x) \o b))$
		      \begin{enumerate}[label=\arabic*. ]
			      \item V
			      \item V
		      \end{enumerate}
		\item $(\no c \y \no y) $
		      \begin{enumerate}[label=\arabic*. ]
			      \item F
			      \item F
		      \end{enumerate}
	\end{enumerate}

\end{multicols}
\[
	\begin{array}{|c|c|c|c|c|c|c|c|}
		\hline
		p & q & p \y q & p \o q & p \sisolosi q \\ \hline
		V & V & V      & V      & V             \\
		V & F & F      & V      & F             \\
		F & V & F      & V      & F             \\
		F & F & F      & F      & V             \\
		\hline
	\end{array}
\]
\ejercicio Determinar, utilizando tablas de verdad,
si las siguientes fórmulas son tautologías,
contradicciones o contingencias.

Nota: Contigencia es que depende de los valores que le de a las proposiciones.

\[
	\begin{array}{|c|c|c|c|c|c|c|c|}
		\hline
		p & q & p \y q & p \o q & p \entonces q & p \sisolosi q \\ \hline
		V & V & V      & V      & V             & V             \\
		V & F & F      & V      & F             & F             \\
		F & V & F      & V      & V             & F             \\
		F & F & F      & F      & V             & V             \\
		\hline
	\end{array}
\]
\begin{enumerate}[label=\alph*)]
	\item $(p\o q)$
	\item
	\item
	\item
	\item
	\item
	\item
	\item
	\item
	\item $(p\entonces(q \entonces r)) \entonces ((p\entonces q) \entonces (p\entonces r)))$ \\
	      \scriptsize
	      $ \begin{array}{|c|c|c|c|c|c|c|c|c|}
			      \hline
			      p & q & r & (q \entonces r ) & (p \entonces (q \entonces r) ) & (p \entonces q) & (p\entonces r) & (p \entonces q) \entonces (p \entonces r)) & (p \entonces (q\entonces r) \entonces (p\entonces q) \entonces (p \entonces r)) \\  \hline
			      V & V & V & V                & V                              & V               & V              & V                                          & V                                                                               \\
			      V & V & F & F                & F                              & V               & F              & F                                          & V                                                                               \\
			      V & F & V & V                & V                              & F               & V              & V                                          & V                                                                               \\
			      V & F & F & V                & V                              & F               & F              & V                                          & V                                                                               \\
			      F & V & V & V                & V                              & V               & V              & V                                          & V                                                                               \\
			      F & V & F & F                & V                              & V               & V              & V                                          & V                                                                               \\
			      F & F & V & V                & V                              & V               & V              & V                                          & V                                                                               \\
			      F & F & F & V                & V                              & V               & V              & V                                          & V                                                                               \\ \hline
		      \end{array} $\\
\end{enumerate}

\ejercicio\FiveStar\   Dadas las proposiciones lógicas $\alpha$ y $\beta$, se dice que $\alpha$ es más fuerte que $\beta$ si y solo si $\alpha \entonces \beta$ es una tautología.
En este caso, también decimos que $\beta$ es más débil que $\alpha$. Determinar la relación de fuerza de los siguientes pares de fórmulas:
\begin{enumerate}[label=\alph*)]
	\item $True$, $False$\\
	      $True \entonces False \sisolosi False \flecha{por lo}[tanto] True$ es más débil que $False$. En particular  $True$ es la fórmula más débil del mundo.

	\item $(p \y q)$, $(p \o q)$.\\
	      Chequeo si $(p \y q) \entonces (p \o q)$ es tautología.\\
	      \[
		      \begin{array}{|c|c|c|c|c|}
			      \hline
			      p & q & (p \y q ) & (p \o q ) & (p \y q) \entonces (p \o q) \\ \hline
			      V & V & V         & V         & V                           \\
			      V & F & F         & V         & V                           \\
			      F & V & F         & V         & V                           \\
			      F & F & F         & F         & V                           \\ \hline
		      \end{array}
	      \]\\
	      Concluyendo que es una tautología, y $(p \y q)$ es más fuerte que $(p \o q)$, la fuerza o $(p \o q)$ es más débil.
	\item $True$, $True$.\\
	      Sale que True es más débil que ella misma y se me explotó la cabeza.\\
	      \red{revisar!}
	\item \hacer
	\item $False$, $False$.\\
	      Sale que False es más débil que ella misma y se me explotó la cabeza.\\
	      \red{revisar!}

	\item \hacer
	\item $p$, $q$. En este caso no hay una tautología.
	      \red{Cómo se responde? Ninguna? o $p$ es más débil que $q$}
	\item \hacer

\end{enumerate}

%5
\ejercicio ¿Cuál es la fórmula proposicional más fuerte y cuál la más débil de las que aparecen en el ejercicio anterior?\\
$True$ y $False$
%6
\ejercicio\FiveStar\ 
\begin{multicols}{2}
	\begin{enumerate}[label=\alph*)]
		\item
		\item
		\item
		\item
		      %e
		\item $(p \y p) \equivale p$ (Idempotencia de la conjunción)\\
		      $
			      \begin{array}{|c|c|c|}
				      \hline
				      p & p & p \y p \\ \hline
				      T & T & T      \\
				      F & F & F      \\ \hline
			      \end{array}
		      $
		\item $(p \o p) \equivale p$ (Idempotencia de la disyunción)\\
		      $
			      \begin{array}{|c|c|c|}
				      \hline
				      p & p & p \o p \\ \hline
				      T & T & T      \\
				      F & F & F      \\ \hline
			      \end{array}
		      $
		\item
		\item
		\item
		\item
		\item
		\item
		\item
		\item
		\item
		\item
		\item
		\item
		\item
		\item
		\item
		\item
		\item
	\end{enumerate}
\end{multicols}

%7
\ejercicio
%8
\ejercicio\FiveStar\ 
%9
\ejercicio
%10
\ejercicio\FiveStar\ 
%11
\ejercicio

%12
\ejercicio

\section{Lógica ternaria o trivalente (Verdadero, Falso o Indefinido)}

%13
\ejercicio\FiveStar\ 
fdsf
%14
\ejercicio\FiveStar\ 
%15
\ejercicio\FiveStar\ 
%16
\ejercicio\FiveStar\ 
%17
\ejercicio\FiveStar\ 

%18
\ejercicio

%19
\ejercicio

\section{Fórmulas del lenguaje de especificación}
%20
\ejercicio\FiveStar\
%21
\ejercicio


% a)
% \begin{itemize}
% 	\item $((p \y p) \y p) \entonces p$
% 	\item True
% 	\item
% 	      $ \begin{array}{|c|c|c|c|c||}
% 			      \hline
% 			      p      & q      & p \ly q & p \lo q & p \lentonces q \\  \hline
% 			      V      & V      & V       & V       & V              \\
% 			      V      & V      & F       & V       & F              \\
% 			      F      & F      & F       & V       & V              \\
% 			      F      & F      & F       & F       & V              \\ \hline
%
% 			      V      & \indef & \indef  & V       & \indef         \\
% 			      F      & \indef & F       & \indef  & V              \\
% 			      \indef & V      & \indef  & \indef  & \indef         \\
% 			      \indef & F      & \indef  & \indef  & \indef         \\
% 			      \indef & V      & \indef  & \indef  & \indef         \\ \hline
% 		      \end{array} $\\
% \end{itemize}
\end{document}

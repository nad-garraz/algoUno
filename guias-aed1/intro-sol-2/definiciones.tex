% Definiciones y nuevos comandos:
% =============

% Conjuntos tipos
\def\reales{\mathbb R}
\def\naturales{\mathbb N}
\def\enteros{\mathbb Z}
\def\complejos{\mathbb C}
\def\i{\text{i}} % i^2 =  -1
\def\equivale{\equiv}

\def\vacio{\varnothing}
\def\union{\cup}
\def\inter{\cap}
\def\existe{\exists\,}
\def\noexiste{\nexists\,}
\def\paratodo{\forall}
\def\talque{\;|\;}
\newcommand{\set}[1] { \left\{ #1 \right\} } % Conjuntos entre llaves

% lógica proposicional
\def\neg{\sim}
\def\y{\land}
\def\o{\lor}
\def\entonces{\rightarrow}
\def\sisolosi{\leftrightarrow}

% lógica trivaluada 
\def\indef{\perp}
\def\no{\,\lnot\,}
\def\lo{\lor_{\text{L}}}
\def\ly{\,\land_{\text{L}}\,}
\def\lentonces{\rightarrow_{\text{L}}}


% problemas
\def\requiere{\texttt{requiere}}
\def\asegura{\texttt{asegura}}
\def\res{\text{\texttt{res}}}
\newcommand{\seq}[1]{\texttt{seq} \langle #1 \rangle}
\newcommand{\secuencia}[1]{\langle#1\rangle}

% =====
% Miscelanea
% =====
\def\estabien{{\color{blue} Consultado, está bien. \checkmark}}
\def\hacer{{\color{black!30!red}Hacer!}}

% separador
\def\separador{\noindent\rule{\linewidth}{0.4pt}\\}
\def\separadorCorto{\noindent\rule{0.5\linewidth}{0.4pt}\\}

% sección ejercicio con su respectivo formato y contador
\newcounter{ejercicio}[section]
\renewcommand{\theejercicio}{\arabic{ejercicio}}
\newcommand{\ejercicio}{%
	\stepcounter{ejercicio}%
	\titleformat{\section}[runin]{\normalfont\bfseries}{Ejercicio \theejercicio}{1em}{}%
	\section*{Ejercicio \theejercicio.}%
}


% Llaves, paréntesis, contenedores
\newcommand{\llave}[2]{ \left\{ \begin{array}{#1} #2 \end{array}\right. }
\newcommand{\llaves}[2]{ \left\{ \begin{array}{#1} #2 \end{array} \right\} }
\newcommand{\matriz}[2]{\left( \begin{array}{#1} #2 \end{array} \right)}


% Colores
\newcommand{\red}[1]{{\color{red}\text{#1}}}
\newcommand{\green}[1]{{\color{olive} \text{#1}}}
\newcommand{\blue}[1]{ {\color{blue} \text{#1}}}
\newcommand{\magenta}[1]{{\color{magenta} \text{#1}}}

% Stackrel text
\newcommand{\stacktext}[2]{ \stackrel{\text{#1}}{#2} }

% Flecha con texto
\NewDocumentCommand{\flecha}{m o}{
	\IfNoValueTF{#2}{
		\xrightarrow[]{\text{#1}}
	}{
		\xrightarrow[\text{#2}]{\text{#1}}
	}
}

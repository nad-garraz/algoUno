\documentclass[12pt,a4paper,spanish]{article}
% Sacar draft para que aparezcan las imagenes.
% Opciones: 10pt, 11pt, landscape, twocolumn, fleqn, leqno...
% Opciones de clase: article, report, letter, beamer...

% Paquetes:
% =========
\usepackage[headheight=110pt, top = 1.5cm, bottom = 1.5cm, left=1.5cm, right=1.5cm]{geometry} %modifico márgenes
\usepackage[T1]{fontenc} % Para tildes en espñol
\usepackage[utf8]{inputenc} % Para poder escribir con tildes en el editor.
\usepackage[english]{babel} % Para que el latex sepa cortar bien las lineas.
\usepackage[ddmmyyyy]{datetime}
\usepackage{amsmath} % Soporte de mathmatics
\usepackage{amssymb} % fuentes de mathmatics
\usepackage{array} % Para tablas y eso
% \usepackage{caption} % Configuracion de figuras y tablas
\usepackage{xcolor} % Para colorear el texto: black, blue, brown, cyan, darkgray, gray, green, lightgray, lime, magenta, olive, orange, pink, purple, red, teal, violet, white, yellow.
\usepackage{graphicx} % Necesario para poner imagenes
\usepackage{enumitem} % Cambiar labels y más flexibilidad para el enumerate
\usepackage{multicol} 
\usepackage{tikz} % para graficar
% \usepackage{fourier} % entre otras cosas me da el \warning symbol
\usepackage{listings} % para codigo

% Definiciones y nuevos comandos:
% =============

% Conjuntos
\def\reales{\mathbb R}
\def\naturales{\mathbb N}
\def\enteros{\mathbb Z}
\def\complejos{\mathbb C}
\def\i{\text{i}} % i^2 =  -1
\def\equivale{\equiv}

\def\vacio{\varnothing}
\def\union{\cup}
\def\inter{\cap}
\def\existe{\exists\,}
\def\noexiste{\nexists\,}
\def\paratodo{\forall}
\def\talque{\;|\;}
\newcommand{\set}[1] { \left\{ #1 \right\} } % Conjuntos entre llaves

% lógica proposicional
\def\neg{\sim}
\def\y{\land}
\def\o{\lor}
\def\entonces{\rightarrow}
\def\sisolosi{\leftrightarrow}

% lógica trivaluada 
\def\indef{\perp}
\def\no{\,\lnot\,}
\def\lo{\lor_{\text{L}}}
\def\ly{\,\land_{\text{L}}\,}
\def\lentonces{\rightarrow_{\text{L}}}

% =====
% Miscelanea
% =====
\newcommand{\estabien}{{\color{blue} Consultado, está bien. \checkmark}}
\newcommand{\hacer}{{\color{black!30!red}Hacer!}}

% separador
\def\separador{\noindent\rule{\linewidth}{0.4pt}\\}
\def\separadorCorto{\noindent\rule{0.5\linewidth}{0.4pt}\\}


% Llaves, paréntesis, contenedores
\newcommand{\llave}[2]{ \left\{ \begin{array}{#1} #2 \end{array}\right. }
\newcommand{\llaves}[2]{ \left\{ \begin{array}{#1} #2 \end{array} \right\} }
\newcommand{\matriz}[2]{\left( \begin{array}{#1} #2 \end{array} \right)}


% Colores
\newcommand{\red}[1]{ {\color{red} \text{#1}}}
\newcommand{\green}[1]{ {\color{olive} \text{#1}}}
\newcommand{\blue}[1]{ {\color{blue} \text{#1}}}
\newcommand{\magenta}[1]{ {\color{magenta} \text{#1}}}

% Stackrel text
\newcommand{\stacktext}[2]{ \stackrel{\text{#1}}{#2} }

% Flecha con texto
\NewDocumentCommand{\flecha}{m o}{
	\IfNoValueTF{#2}{
		\xrightarrow[]{\text{#1}}
	}{
		\xrightarrow[\text{#2}]{\text{#1}}
	}
}

% sección ejercicio con su respectivo formato y contador
\newcounter{ejercicio}[part]
\renewcommand{\theejercicio}{\arabic{ejercicio}}
\newcommand{\ejercicio}{%
	\stepcounter{ejercicio}%
	\titleformat{\subsubsection}[runin]{\normalfont\bfseries}{Ejercicio \theejercicio}{-1em}{}%
  \subsubsection*{Ejercicio \theejercicio.}%
}
	\let\oldsection\part
	\renewcommand{\part}{\stepcounter{ejercicio}\part}
 % idem con las definiciones

\usepackage{titlesec} % para editar titulos y hacer secciones con formato a medida

\begin{document}

\pagestyle{empty} % enumeración de páginas

% Datos de importancia
\title{Práctica 2 de Intro a la programación}
\author{Nad Garraz}
\date{Actualizado: \today} % Cambiar de ser necesario
\maketitle
\thispagestyle{empty}


\section{Formato de especificaciones semiformales:}

Nota: Contrato $\entonces$ especificaciones. Contrato entre 2 partes.
\begin{itemize}
	\item Contrato: el programador escribe un programa P tal que si el usuario suministra datos que hacen verdadera
	      la precondición, entonnces P termian en una cantidad finita de pasos retornando un valor que hace verdadera la postcondición.
	\item El programa P es \red{correcto} para la especificación dada por la precondición y la postcondición exactamente cuando se cumple el contrato.
	\item Si el usuario no cumple la precondición y P se cuelga o no cumple la postcondición...
	      \begin{itemize}
		      \item ¿El usuario tiene derecho a quejarse? \red{Creería que no.}
		      \item ¿Se cumple el contrato? \red{No, se debe cumplir precondición para que se cumpla el contrato.}
	      \end{itemize}
	\item Si el usuario cumple la precondición y P se cuelga o no cumple la postcondición...
	      \begin{itemize}
		      \item ¿El usuario tiene derecho a quejarse? \red{Diría que sí.}
		      \item ¿Se cumple el contrato? \red{No, porque debería terminar el programa cumpliendo la postcondición.}
	      \end{itemize}
\end{itemize}

\separador
%1
\ejercicio

\begin{enumerate}[label=\alph*)]
	%a
	\item
	      \[
		      \begin{array}{|c|c|}
			      \hline
			      \texttt{entrada }x & \texttt{salida }res \\  \hline
			      0                  & 2                   \\
			      2                  & 4                   \\
			      -2                 & -4                  \\\hline
		      \end{array}
	      \]

	      %b
	\item
	      \requiere: \{$x$ es positivo\}\\
	      \asegura: \{$resultado$ es la raíz cuadrada de $x$\}
	      \[
		      \begin{array}{|c|c|}
			      \hline
			      \texttt{entrada }x & \texttt{salida }res \\  \hline
			      1                  & 1                   \\
			      \red{$4$}          & \red{$-2$}          \\
			      25                 & 5                   \\\hline
		      \end{array}
	      \]
	      \red{Consultar por el tema de la definición de raíz cuadrada en este ejercicio}\\
	      \blue{es justamente la definicion que está en el inciso (b.). Es medio circular el asunto.}\\
	      \blue{Supongo que está bien definida de ahora en más. Y que no vale que la $res < 0$}


	      %c
	\item
	      \def\salidaLarga{
		      \begin{array}{c}
			      \red{2} \\
			      \red{1}
		      \end{array}
	      }
	      \[
		      \begin{array}{|c|c|}
			      \hline
			      \texttt{entrada }x & \texttt{salida }res \\ \hline
			      -1                 & 1                   \\
			      1.4                & 1                   \\
			      1.6                & 2                   \\ \hline
			      \red{1.5}          & \salidaLarga        \\ \hline
		      \end{array}
	      \]
	      \red{Consultar por el tema del redondeo. Al estar igual distancia el 1 y 2, se cumple la poscondición?}\\
	      \red{Puedo tener ese tipo de ambigüedad?}\\
	      \blue{Al parecer ya sea $1$ o $2$, funcionaría como respueta. Dado que cumplen el asegura.}

	      %d
	\item
	      \[
		      \begin{array}{|c|c|}
			      \hline
			      \texttt{entrada } s & \texttt{salida } res       \\  \hline
			      \secuencia{2, 4, 9} & \secuencia{\sqrt{2}, 2, 3} \\
			      \secuencia{25}      & \secuencia{5}              \\\hline
		      \end{array}
	      \]

	      %e
	\item
	      \def\salidaLarga{
		      \begin{array}{c}
			      \secuencia{2, \sqrt{2}, 3} \\
			      \secuencia{2, 3, \sqrt{2}} \\
			      \secuencia{3, 2, \sqrt{2}} \\
			      \secuencia{3, \sqrt{2}, 2} \\
			      \secuencia{\sqrt{2}, 2, 3} \\
			      \secuencia{\sqrt{2},3, 2}  \\
		      \end{array}
	      }
	      \[
		      \begin{array}{|c|c|}
			      \hline
			      \texttt{entrada } s & \texttt{salida } res \\  \hline
			      \secuencia{2, 4, 9} & \salidaLarga         \\ \hline
			      \secuencia{25}      & \secuencia{5}        \\ \hline
		      \end{array}
	      \]

	      %f
	\item
	      \def\salidaLarga{
		      \begin{array}{c}
			      \secuencia{2, 2, 2}                      \\
			      \secuencia{\sqrt{2}, \sqrt{2}, \sqrt{2}} \\
			      \secuencia{3, 3, 3}                      \\
			      \secuencia{2,9, 2}                       \\
			      \secuencia{3, 2, 4}                      \\
			      \secuencia{\sqrt{2}, 2, 9}               \\
			      \secuencia{\sqrt{2},9, 4}                \\
		      \end{array}
	      }
	      \[
		      \begin{array}{|c|c|}
			      \hline
			      \texttt{entrada } s & \texttt{salida } res \\  \hline
			      \secuencia{2, 4, 9} & \salidaLarga         \\ \hline
			      \secuencia{25}      & \secuencia{5}        \\ \hline
		      \end{array}
	      \]

	      %g
	\item

	      \def\salidaLarga{
		      \begin{array}{c}
			      \secuencia{2, 3}       \\
			      \secuencia{3, 2}       \\
			      \secuencia{2, 3, 2, 3} \\
		      \end{array}
	      }
	      \[
		      \begin{array}{|c|c|}
			      \hline
			      \texttt{entrada } s  & \texttt{salida } res \\  \hline
			      \secuencia{-2, 4, 9} & \salidaLarga         \\ \hline
			      \secuencia{25}       & \secuencia{5}        \\ \hline
		      \end{array}
	      \]

	      %h
	\item

	      \def\salidaLarga{
		      \begin{array}{c}
			      \secuencia{\sqrt{2}, 2, 3}            \\
			      \secuencia{\sqrt{2}, 2, 3, e^{i2\pi}} \\
		      \end{array}
	      }
	      \[
		      \begin{array}{|c|c|}
			      \hline
			      \texttt{entrada } s & \texttt{salida } res \\  \hline
			      \secuencia{2, 4, 9} & \salidaLarga         \\ \hline
			      \secuencia{25}      & \secuencia{5}        \\ \hline
		      \end{array}
	      \]

	      %i
	\item

	      \def\salidaLarga{
		      \begin{array}{c}
			      \secuencia{\sqrt{2}, 2, 3} \\
		      \end{array}
	      }
	      \[
		      \begin{array}{|c|c|}
			      \hline
			      \texttt{entrada } s    & \texttt{salida } res \\  \hline
			      \secuencia{2, 4, 9}    & \salidaLarga         \\ \hline
			      \secuencia{25}         & \secuencia{5}        \\ \hline
			      \secuencia{\sqrt{3},4} & \secuencia{100, 2}   \\ \hline
		      \end{array}
	      \]
\end{enumerate}

%2
\ejercicio

\begin{enumerate}[label=\arabic*.]
	\item
	      En los problemas raicesCuadradas que utilizan el problema raizCuadrada, ¿Se puede eliminar el requiere:
	      \textit{"Todos los elementos de s son positivos"}? Justificar.  \\
	      \separadorCorto
	      No. Porque necesitás $devolver$ un número $\reales$.

	\item
	      ¿Qué consecuencia tiene la diferencia de $asegura$ en el resultado entre los problemas\\
	      \texttt{raicesCuadradasUno} y\texttt{raicesCuadradasDos}?  \\
	      \separadorCorto
	      Se ve en la respuesta dada al ejercicio. Se puede cambiar la posición de los resultados en la $\seq{\reales}$

	\item
	      En base a la respuesta del ítem anterior, ¿Un algoritmo que satisface la especificación de \texttt{raicesCuadradasUno}, también
	      satisface la especificación de \texttt{raicesCuadradasDos}? ¿Y al revés? \\
	      \separadorCorto
	      Sí, el conjunto solución de \texttt{raicesCuadradasUno}, es más pequeño, está contenido en la solución de \texttt{raicesCuadradasDos},
	      es decir:\\
	      Las condiciones de los asegura del \texttt{res(raicesCuadradasUno)} son \textbf{más fuertes} que las de \texttt{res(raicesCuadradasDos)}.
	      \begin{align*}
		      \set{\seq{ \texttt{res(raicesCuadradasUno)}}} & \subseteq \set{\seq{\texttt{res(raicesCuadradasDos)}}} \\
		      \texttt{res(raicesCuadradasUno)}              & \entonces \texttt{res(raicesCuadradasDos)}             \\
		      \textbf{fuerte}                               & \entonces \textbf{débil}
	      \end{align*}

	      Las condiciones fuertes dan en general un conjunto de soluciones más chico. Las condiciones débiles son más relajadas, por lo que permiten
	      que se meta mucha basura no deseada en el conjunto solución.

	\item
	      Explicar en palabras las diferencias entre los problemas \texttt{raicesCuadradasCinco} y \\ \texttt{raicesCuadradasSeis}.
	      ¿Cómo influye el asegura de longitud máxima? ¿Es $\secuencia{ \sqrt{3}, \sqrt{9}}$ una salida válida para ambos problemas,
	      dado $s = \secuencia{ 3, 9, 11, 15, 18}$ ? ¿Es $\secuencia{ \sqrt{3}, \sqrt{9},\sqrt{11},\sqrt{13}}$ una salida válida para el problema
	      \texttt{raicesCuadradasCinco} dado $s = \secuencia{ 3, 9, 11 }$ ambos problemas,\\
	      \separadorCorto
	      Los \asegura son más fuertes en \texttt{raicesCuadradasSeis}, al evitar poner elementos que nada tengan que ver con $s$ limitando la longitud
	      de $res$. \asegura fuerte $\entonces$ \asegura débil, por lo que la solución:
	      $\secuencia{\sqrt{3}, \sqrt{9}}$ es solución de ambos, sin embargo $\secuencia{\sqrt{3}, \sqrt{9},\sqrt{11},\sqrt{13}}$ no es solución
	      del \texttt{problema} con los \asegura fuertes, porque no cumple que sea solución de $s = \secuencia{ 3, 9, 11 }$ violando el \asegura
	      sobre la longitud $res \leq s$

	\item ¿Cómo me cambia en el \texttt{problema raicesCuadradasCuatro} agregar un \asegura que diga que $res$ tiene misma longitud que $s$? \\
	      \separadorCorto
	      Eso me  obligaría a tener la respueta de \texttt{problema raicesCuadradasDos}. Si bien no puedo agregar basura, puedo intercambiar los índices
	      de $res$

	\item ¿Si los problemas \texttt{raicesCuadradasDos} y \texttt{raicesCuadradasTres} tienen el mismo $res$ para la misma entrada (una secuencia
	      específica de números), quiere decir que son el mismo problema?\\
	      \separadorCorto
	      No, el \asegura del \texttt{raicesCuadradasDos} es más fuerte que el del \texttt{raicesCuadradasTres}.

	\item ¿Qué ocurre si eliminamos los \requiere "no hay repetidos"?¿Es $\secuencia{ 2, 2, 1 }$ una salida válida
	      para el problema raicesCuadradasDos dado $s = \secuencia{4, 1, 1}$?\\

	      \separadorCorto

	      El \texttt{problema} quedaría así:\\
	      \requiere: \{Todos los elementos de $s$ son positivos\}\\
	      \asegura: \{$res$ tiene la misma cantidad de elementos que $s$.\}\\
	      \asegura: \{Los elementos de $res$ son la salida de aplicar el problema \texttt{raizCuadrada()} a todos los elementos de la secuensia $s$\}\\

	      Sí, es una respuesta válida, dado que $\secuencia{2, 2, 1}$ tiene igual longitud que $\secuencia{4,1,1}$ y los valores son las raíces cuadradas
	      de los elementos de $s$.\\
	      \red{¿Elementos de $s$ con igual valor pero en índices distintos, son el mismo elemento?}\\
	      \red{¿Es como en álgebra? $\boxed{\secuencia{1,1,2} \stackrel{?}= \secuencia{1,2,2}}$ }

\end{enumerate}

%3
\ejercicio

Responder las preguntas dada la siguiente especificación para el problema de ordenar una secuencia de enteros (es decir, dada una secuencia de enteros,
devolver los mismos elementos ordenados de menor a mayor):\\

\begin{verbatim}
problema ordenar (s: seq<Z>): seq<Z> {
  requiere: {True}
  asegura: {resultado es una secuencia de la cual cada elemento es
            estrictamente mayor que el anterior}
} 
\end{verbatim}

\begin{enumerate}[label=\alph*)]
	\item  Dado $s = \secuencia{4,3,5}$ como secuencia de entrada ¿Es válida $res = \secuencia{3,4,5}$. $\to$ sí.
	\item \hacer
	\item \hacer
	\item \hacer
	\item \hacer
	\item \hacer
\end{enumerate}

%4
\ejercicio

Se desea especificar el problema de duplicar todos los valores de la secuencia y se cuenta con la siguiente especificación:

\begin{verbatim}
problema duplicarTodos (s: seq<Z>): seq<Z> {
  requiere: {True}
  asegura: {resultado tiene la misma cantidad de elmentos que s}
} 
\end{verbatim}

\begin{enumerate}[label=\alph*)]
	\item  ¿Qué problemas tiene la especificaión dada? Dar ejemplos de valores para $resultado$ que satisfagan la especificación pero no sean respuestas correctas.\\
	      \hacer
	\item  Indicar cuál/es de los siguientes \asegura debería/n ser agregado/s a la especificación. Justificar en cada caso por qué deberían o no ser agregados.
	      \begin{itemize}
		      \item \asegura: \{Para cada valor $x$ que pertenece a $s$, hay algún valor en $resultado$ que es la salida de duplicar($x$).\}
		      \item \asegura: \{En cada posición de $resultado$, el valor es mayor al valor en esa misma posición de $s$.\}
		      \item \asegura: \{En cada posición de $resultado$, el valor es igual a la salida de aplicar \texttt{duplicar()} al valor en esa misma posición de $s$.\}
		      \item \asegura: \{Todos los elementos de $resultado$ son números pares.\}
	      \end{itemize}
	      \hacer
	\item
\end{enumerate}



a es válida
b  no cumple el $<$ ... se corrige con $\leq$ o achicar los requiere, por ejemplo, no tener elementos repetidos.

f) problema ordenar (s: seq($\enteros$): seq($\enteros$))

requiere: {True}
asegura: {resultado es una secuencia con todos los mismos elementos que
$s$}
asegura: {\#(A,s) = \#(A.resultado)}
asegura: {resultado tiene igual long que s }
asegura: {ordenados de menor a mayor}


5)
problema cantidadColectivosLinea(linea: $\enteros$, colectivos: $\seq{\enteros}$, colectivosHoy: $\seq{\enteros}$ : $\enteros$){
requiere: {linea $\in$ colectivos}
asegura: {respuesta \#(linea, colectivos)}

}


1d
1e
2.2
2.3
1h
1i
2.4
2.7
3 a,b,c,d,e,f
5 a

\end{document}

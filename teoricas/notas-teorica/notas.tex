\documentclass[12pt,a4paper, spanish]{article}
% Sacar draft para que aparezcan las imagenes.
% Opciones: 10pt, 11pt, landscape, twocolumn, fleqn, leqno...
% Opciones de clase: article, report, letter, beamer...

% Paquetes:
% =========
\usepackage[headheight=110pt, top = 2cm, bottom = 2cm, left=1cm, right=1cm]{geometry} %modifico márgenes
\usepackage[T1]{fontenc} % tildes
\usepackage[utf8]{inputenc} % Para poder escribir con tildes en el editor.
\usepackage[english]{babel} % Para cortar las palabras en silabas, creo.
\usepackage[ddmmyyyy]{datetime}
\usepackage{amsmath} % Soporte de mathmatics
\usepackage{amssymb} % fuentes de mathmatics
\usepackage{array} % Para tablas y eso
\usepackage{caption} % Configuracion de figuras y tablas
\usepackage[dvipsnames]{xcolor} % Para colorear el texto: black, blue, brown, cyan, darkgray, gray, green, lightgray, lime, magenta, olive, orange, pink, purple, red, teal, violet, white, yellow.
\usepackage{graphicx} % Necesario para poner imagenes
\usepackage{enumitem} % Cambiar labels y más flexibilidad para el enumerate
\usepackage{multicol} 
\usepackage{tikz} % para graficar
\usepackage{cancel}

% para hacer los graficos tipo grafos
\usetikzlibrary{shapes,arrows.meta, chains, matrix, calc, trees, positioning, fit}
\usetikzlibrary{external}

% Definiciones y nuevos comandos:
% =============

% Conjuntos tipos
\def\reales{\mathbb R}
\def\naturales{\mathbb N}
\def\enteros{\mathbb Z}
\def\complejos{\mathbb C}
\def\i{\text{i}} % i^2 =  -1
\def\equivale{\equiv}

\def\vacio{\varnothing}
\def\union{\cup}
\def\inter{\cap}
\def\existe{\exists\,}
\def\noexiste{\nexists\,}
\def\paratodo{\forall}
\def\talque{\;|\;}
\newcommand{\set}[1] { \left\{ #1 \right\} } % Conjuntos entre llaves

% lógica proposicional
\def\neg{\sim}
\def\y{\land}
\def\o{\lor}
\def\entonces{\rightarrow}
\def\sisolosi{\leftrightarrow}

% lógica trivaluada 
\def\indef{\perp}
\def\no{\,\lnot\,}
\def\lo{\lor_{\text{L}}}
\def\ly{\,\land_{\text{L}}\,}
\def\lentonces{\rightarrow_{\text{L}}}


% problemas
\def\requiere{\texttt{requiere}}
\def\asegura{\texttt{asegura}}
\def\res{\text{\texttt{res}}}
\newcommand{\seq}[1]{\texttt{seq} \langle #1 \rangle}
\newcommand{\secuencia}[1]{\langle#1\rangle}

% =====
% Miscelanea
% =====
\def\estabien{{\color{blue} Consultado, está bien. \checkmark}}
\def\hacer{{\color{black!30!red}Hacer!}}

% separador
\def\separador{\noindent\rule{\linewidth}{0.4pt}\\}
\def\separadorCorto{\noindent\rule{0.5\linewidth}{0.4pt}\\}

% sección ejercicio con su respectivo formato y contador
\newcounter{ejercicio}[section]
\renewcommand{\theejercicio}{\arabic{ejercicio}}
\newcommand{\ejercicio}{%
	\stepcounter{ejercicio}%
	\titleformat{\section}[runin]{\normalfont\bfseries}{Ejercicio \theejercicio}{1em}{}%
	\section*{Ejercicio \theejercicio.}%
}


% Llaves, paréntesis, contenedores
\newcommand{\llave}[2]{ \left\{ \begin{array}{#1} #2 \end{array}\right. }
\newcommand{\llaves}[2]{ \left\{ \begin{array}{#1} #2 \end{array} \right\} }
\newcommand{\matriz}[2]{\left( \begin{array}{#1} #2 \end{array} \right)}


% Colores
\newcommand{\red}[1]{{\color{red}\text{#1}}}
\newcommand{\green}[1]{{\color{olive} \text{#1}}}
\newcommand{\blue}[1]{ {\color{blue} \text{#1}}}
\newcommand{\magenta}[1]{{\color{magenta} \text{#1}}}

% Stackrel text
\newcommand{\stacktext}[2]{ \stackrel{\text{#1}}{#2} }

% Flecha con texto
\NewDocumentCommand{\flecha}{m o}{
	\IfNoValueTF{#2}{
		\xrightarrow[]{\text{#1}}
	}{
		\xrightarrow[\text{#2}]{\text{#1}}
	}
}


\begin{document}
% Datos de importancia
\title{Notas de la teórica}
\author{Nad Garraz}
\date{\today} % Cambiar de ser necesario

Nota: Se empieza por especificar un problema. Luego se procede a escribir \textbf{un} \red{algoritmo} que cumpla la especifcación. Una vez
escrito el \red{algoritmo} se escribe el \red{programa} que implementa el \red{algoritmo}.
\begin{itemize}
	\item Tipo de datos:
	      \begin{itemize}
		      \item Enteros $\enteros$
		      \item Reales $\reales$
		      \item Bool $\mathbb{B} = \set{\mathbf{true, false}}$
		      \item Char : Hay función char(z:$\enteros$) y ord(c:Char) $c: char, \ordC{c} + 1 = \ordC{d}, char()$
          \item Enum: \blue{\texttt{enum} \textit{Nombre} $\set{\mathtt{CONSTANTE1, CONSTANTE2, CONSTANTE3, \dots}}$}.\\
		            $\llave{l}{
                  \ordE{CONSTANTE1} = 0\\
                     Nombre(2) = \mathtt{CONSTANTE3}
			            }$
                \item Upla o tupla: $\tipo_0 \times \tipo_1 \times \cdots \times \tipo_k$.\\
                  $\llave{l}{
                    (7,5)_0 = 7\\
                    ('a', \mathtt{DOM}, 78)_2 = 78
                  }$
	      \end{itemize}
	\item especificación
	      \begin{itemize}
		      \item nombre
		      \item parámetros
		      \item tipo de dato del resultado
		      \item etiquetas: opcionales en los \requiere y \asegura
	      \end{itemize}
	\item
	      Funciones sobre secuencias:
	      \begin{enumerate}
		      \item Longitud, \lengthF(a), $|a|$, $a.\lengthF$

		      \item Indexación: $\seqF{} [i:\enteros] : \tipo$. Se nota: $a[i]$

		      \item Pertenece: $\perteneceF(x: \tipo,\, s: \seq\ang{\tipo}): Bool$. Se nota: $\perteneceF(x,s)$ o $x \in s$

		      \item Igualdad: $\seq{\tipo} = \seq{\tipo} $

		      \item Cabeza: $\headF(a: \seq{\tipo}): \tipo$

		      \item Cola: $\tailF(a: \seq{\tipo}): \seq{\tipo}$

		      \item Agregar cabeza: $\addFirstF(t: \tipo, a: \seq{\tipo}): \seq{\tipo}$

		      \item Concatenación: $\concatF(a: \seq{\tipo}, b: \seq{\tipo}): \seq{\tipo} $ (notación: $a++b$)

		      \item Subsecuencia: $\subseqF(a: \seq{\tipo}, d, h: \enteros): \seq{\tipo} $

		      \item Cambiar una posición: $\setAtF(a: \seq{\tipo}, i: \enteros, val: \tipo): \seq{\tipo} $
	      \end{enumerate}

	\item Modularización:
	      \begin{itemize}
		      \item Descomponer un problema grande en otros más chicos.
		      \item Componerlos y obtener la solución al problema original.
	      \end{itemize}
	      Esto favorece muchos aspectos de calidad como:
	      \begin{itemize}
		      \item Reutilización, una función auxiliar puede ser utilizada en muchos contextos)
		      \item Es más fácil probar algo chino que algo grande.
		      \item La declaratividad, es más fácil de entender.
	      \end{itemize}

      \item Programa Funcional
        \begin{itemize}
          \item ecuaciones orientadas
          \item evaluación de una expresión
          \item Transparencia referencial. El contexto no afecta a la función. Permite demostrar correctitud de forma modular.
          \item Formación de expresiones:
          \begin{itemize}
            \item atómicas o formas normales. Irreducibles. $\sim$ valores. Ej: 2, false, (3, true)
            \item compuestas. Se forman con expresiones atómicas y operaciones. Ej: 1+1, (4-1, true || false)
          \end{itemize}
        \end{itemize}
\end{itemize}


\end{document}
